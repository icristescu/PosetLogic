\begin{definition}[Event]
  Define an event $e=(r:L\action R,m_k:L\lemb M)$ as a pair consisting of a rule $r:L\action R$ and a \emph{matching} $m_k:L\lemb M$ of the left hand side of $r$ in a site graph $M$. We denote $\labl(e) = r$.
\end{definition}

\begin{property}
  $e\dashv e' \implies\labl(e)\xrightarrow{-}\labl(e')$.
\end{property}
\begin{proof}
  \begin{mdframed}[backgroundcolor=blue!20]
    to do
  \end{mdframed}
\end{proof}

\begin{property}
\label{prop:cause_pos_infl}
  $e< e'\implies \labl(e)\xrightarrow{+}\labl(e')$.
\end{property}
\begin{proof}
  \begin{mdframed}[backgroundcolor=blue!20]
    to do
  \end{mdframed}
\end{proof}

Note that if $e_1\ll e_2$ there are three cases possible:
\begin{itemize}
\item $e_1< e_2$
\item $\labl(e)\xrightarrow{+}\labl(e')$ and $\neg(e_1 < e_2)$
\item $\neg(\labl(e)\xrightarrow{+}\labl(e'))$
\end{itemize}

%%%%%%%%%%%%%%% on influence due to too much context %%%%%%%%%%%%%%%
This technique introduces context into the rules. This context might interfere with the intuitive reading of influence.

\begin{example}[Influence on refined rules]
\label{ex:infl_refined}
Let us consider the following two rules:

\begin{verbatim}
r1:  A,B -> A-B
r2:  C,D -> D
\end{verbatim}

Let us additionally consider that rule \verb|r1| occurs in a poset and is refined to \verb|A,B,C -> A-B,C|. We then observe a negative influence of \verb|r2| to \verb|r1|.
\end{example}

If we do not want a negative influence in~\autoref{ex:infl_refined}, we need to refine the definition of influence such that the domain of definition in both rules is considered.

Unrefined rules suggest a local modification on a graph, whereas the refined one are modifications in a larger context. Thus we need to keep in mind the context of the refined rule in order not to overapproximate the influence.

\begin{definition}[Negative influence in a context]
  \label{def:low_res}
  Let $L_1{\remb} D_1 {\lemb} R_1$ and $L_2{\remb} D_2 {\lemb} R_2$ be two rules such that the square $O\lemb R_1\lemb M\remb L_2\remb O$ is induced by a negative influence $r_1\redl{-}_o r_2$.

The negative influence is not an overapproximation, if the pullback of the cospan $D_2\lemb R_1\remb O$, denoted $P_2$, and $O$ is not an iso. In other words, $O$ is not contained in $P_2$.
  \[
  \begin{tikzpicture} %[scale=0.8]
    \node (p1) at (0,-1) {\(P_1\)};
    \node (p2) at (2,-1) {\(P_2\)};
    \node (d1) at (-1,0) {\(D_1\)};
    \node (d2) at (3,0) {\(D_2\)};
    \node (o) at (1,0) {\(O\)};
    \node (m) at (1,2) {\(M\)};
    \node (r1) at (0,1) {\(L_1\)};
    \node (l1) at (-2,1) {\(R_1\)};
    \node (l2) at (2,1) {\(L_2\)};
    \node (r2) at (4,1) {\(R_2\)};
    \draw [right hook->] (p1) -- (d1);
    \draw [left hook->] (p1) -- node [right,midway] {$\not\iso$}  (o);
    \draw [right hook->] (p2) -- (d2);
    \draw [left hook->] (p2) -- node [right,midway] {$\not\iso$}  (o);
    \draw [right hook->] (d1) -- (r1);
    \draw [right hook->] (d1) -- (l1);
    \draw [right hook->] (d2) -- (r2);
    \draw [right hook->] (d2) -- (l2);
    \draw [left hook->] (o) -- (r1);
    \draw [left hook->] (o) -- (l2);
    \draw [left hook->] (r1) --  (m);
    \draw [left hook->] (l2) --  (m);
  \end{tikzpicture}
  \]

\end{definition}
