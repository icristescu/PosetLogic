\section{Logic on posets}

\begin{definition}[Poset]
  \label{def:poset}
  A poset $s=\{E,<\}$ is a set of events $E$ and a binary relations on events $< \subseteq E\times E$. We denote $\tleq$ the transitive and reflexive closure of $<$.
  We use the predicates $\text{min}(s)$ and $\text{max}(s)$ to retrieve the minimal and maximal events, respectively.

  A labelled poset $s=\{E,<,\labl\}$ is a set with an additional labelling function $\labl$ defined on events.
\end{definition}

%\begin{definition}[Immediate causality]
%  Let us define immediate causality $e<_s e'$, for two events $e,e'\in s$ such that $e\sqsubset e'$ and $\nexists e''$, $e\sqsubset e''\sqsubset e'$.
%\end{definition}

%We use the notations $s \subseteq s'$, $s\cap s'$, $s\cup s'$ and $s\iso s'$ for the respective set operations that are respectful of the causal order and labelling function. - just in kappa

Let $R$ a set of rules (labels on events) and $\mathcal{S}$ a set of posets.
%closed under set inclusion: $s\in \mathcal{S} \implies \forall s'\subset_{\labl} s, s'\in\mathcal{S}$.
We denote $A,B$ elements of $R$ and $s$ elements of $\mathcal{S}$.

\subsection{Grammar}

\begin{align*}
  x ::= & x^e ~|~ x^s & \tag{variables on events and stories} \\
  t^s ::= & x^s ~|~ s ~|~ \text{min}(t^s) ~|~ \text{max}(t^s) ~|~ t^s_1 \cup t^s_2 ~|~ t^s_1 \cap t^s_2 &\tag{terms on posets} \\
  t^e ::= & x^e ~|~ e & \tag{terms on events}\\
  t ::= & t^s ~|~ t^e & \tag{terms}\\
  \\
  \varphi ::= & \exists x.\varphi(x) ~|~ \forall x.\varphi(x) ~|&\tag{quantifiers}\\
  & \neg \varphi ~|~\varphi_1 \wedge \varphi_2~|&\tag{logical connectors}\\
  & t^e\in t^s ~|~ \labl(t^e) = A ~|~ t^e_1 <_{t^s} t^e_2 ~|~ t^e_1 \tleq_{t^s} t^e_2 ~|~ t^e_1\in t^s_1 \dashv t^e_2\in t^s_2~|~ t^s_1 = t^s_2 ~|~ t^s_1\subseteq t^s_2
  & \tag{predicates}\\
\end{align*}

We define the logical connectors $\vee$, $\implies$ in the usual manner.

\subsection{Interpretation}

We interpret the logic on the domain of events and posets.
%an interpretation is the link between syntax and semantics. The functions and predicates are interpreted as their corresponding operations on events and posets.

All function and predicates used in our logic have standard interpretation on posets, with maybe the exception of $\dashv$. We can interpret it as any predicates that relates events in different posets. Our interpretation in the restrainted case of graph rewriting systems (and in Kappa) will be that of inhibition (\autoref{def:ref_neg_infl}).

A \emph{valuation} for $\varphi$ is a function
$v:\text{fv}(\varphi)\to\mathcal{E}\uplus\mathcal{S}$
from the set of free variables of $\varphi$ to the set of events and posets.
%
The evaluation of $\varphi$ is defined below, where a valuation function is needed for the set of free variables of $\varphi$.
%We use two functions $\enct$ to evaluate terms and $\enc$ to evaluate formulas.
\begin{align*}
  \enc{\forall x^s.\varphi}_{v} &= T \iff\text{ for all }s\in\mathcal{S}, \enc{\varphi(s/x)}_{v} = T\\
  \enc{\exists x^s.\varphi}_{v} &= T \iff\text{ for some }s\in\mathcal{S}, \enc{\varphi(s/x)}_{v} = T\\
  \enc{\neg\varphi}_v &= \neg\enc{\varphi} \\
  \enc{\varphi_1\wedge\varphi_2}_v &= \enc{\varphi_1}_v\wedge\enc{\varphi_2}_v\\
  \enc{t^e\in t^s}_v &= T\iff\enct{t^e}_v\in\enct{t^s}_v\\
  \enc{\labl(t^e) = A}_v &= T\iff\labl(\enct{t^e}_v)=A\\
  \enc{t^e_1 <_{t^s} t^e_2} &= T\iff e_1 <_s e_2\text{ where }e_1 = \enc{t^e_1}_v,e_2 = \enc{t^e_2}_v,s = \enc{t^s}_v\\
  \enc{t^e_1 \tleq_{t^s} t^e_2} &= T\iff e_1 \tleq_s e_2\text{ where }e_1 = \enc{t^e_1}_v,e_2 = \enc{t^e_2}_v,s = \enc{t^s}_v\\
  \enc{t^e_1\in t^s_1 \dashv t^e_2\in t^s_2}_v &= T\iff e_1\in s_1 \dashv e_2\in s_2\text{ where }
  e_1 = \enc{t^e_1}_v,e_2 = \enc{t^e_2}_v,s_1 = \enc{t^s_1}_v,s_2 = \enc{t^s_2}_v\\
  \enc{t^s_1\subseteq t^s_2}_v &= T\iff\enc{t^s_1}_v\subseteq \enc{t^s_2}_v\\
  \\
  \enct{x}_{v} &= v(x)\\
  \enct{e}_{v} &= e\\
  \enct{s}_{v} &= s\\
  \enct{\text{min}(t^s)}_{v} &= \text{min}(\enct{t^s})\\
  \enct{\text{max}(t^s)}_{v} &= \text{max}(\enct{t^s})\\
  \enct{t^s_1 \cup t^s_2}_{v} &= \enct{t^s_1}_v \cup \enct{t^s_2}_{v}\\
  \enct{t^s_1 \cap t^s_2}_{v} &= \enct{t^s_1}_v \cap \enct{t^s_2}_{v}\\
\end{align*}
%pred stands for predicates and func for the functions used in the grammar of $t^s$.

A formula $\varphi$ is satisfiable if there exists $v$ such that $\enc{\varphi}_v = T$.

The denotation of $\varphi$, denoted $\den{\varphi}$ is the set of valuations for which $\varphi$ evaluates to true. We give the denotation of a formula in~\autoref{app:denote}.
